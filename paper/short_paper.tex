\documentclass{article}

\usepackage{times}
\usepackage{hyperref}

% Figures
% 
% Fig 1 -- Explanation of the metric: schematic diagram of paths and high-degree
% vs low-degree nodes. Possibly extracted from the data. Take a look at diagram in
% Leskovec paper on Wikipedia paths (WWW'12?).
% 
% Fig 2 -- Panel 1: local structure of politician-ideology network (with paths)
% Panel 2: DWNOMINATE comparison plot. Table with F1, AUC for random forests and
% nearest neighbors on all combos of metric/weight/{directed,undirected}
% 
% Fig 3 -- Four confusion matrices: oscars, capitals, presidents, countries
% 
% Fig 4 -- Relation extraction (only use DEGREE data): Panel 1: diagram with
% snapshot from Wikipedia page text with sentence highlighted, mapped entities and
% actual ratings. Panel 2: ROC curve plot (closure with edge removal, noclosure).
% 
% Quote rank order correlation (Kendall/Spearman) in text or as a table (panel 3
% in Fig4 )

\author{
    Giovanni Luca Ciampaglia\thanks{Corresponding author: gciampag@indiana.edu} \\
    Prashant Shiralkar \\
    Johan Bollen \\
    Luis M. Rocha \\
    Alessandro Flammini \\
    Filippo Menczer \\[1em]
    {\small School of Informatics and Computing, Indiana University }\\
    {\small 919 E 10\textsuperscript{th}, Bloomington, 47408 IN, USA}
}

\title{Can Machines Determine Truth?}
\date{}
 
\begin{document}

\maketitle

\begin{abstract}
	This is the abstract.
\end{abstract}

\section*{Paper outline}

\begin{enumerate}

    \item The journalistic practice of fact-checking -- determining the
        truthfulness of non-fictional statements -- is a matter of increasing
        importance in today's information-rich world. Politics, public opinion,
        public health, natural and man-made disasters, are all environments in
        which the access to accurate, unbiased information is of vital
        importance for the correct working of the system and in which the
        spreading of inaccurate or malicious information can cost lives and
        invaluable assets.

    \item Traditional fact-checking is a painstaking activity. Moreover,
        \cite{VariousAuthors2010}
        

\end{enumerate}

\section{Introduction}

\section{Related Work}

Blah

\section{Methods}

This is the methods section.

\subsection{DBpedia knowledge network}

Here we talk about DBPedia.

\subsection{Epistemic closure}

Here we talk about epistemic closures: metrics, distance, weights, undir/dir.

\section{Results}

\subsection{Calibration}

\subsection{Validation}

\subsection{Examples from Social media}

\section{Conclusion}

\bibliographystyle{plain}
\bibliography{biblio}
\end{document}

% vim: set sts=4 sw=4 expandtab nowrap:
